\documentclass[a4paper,11pt]{article}

\usepackage[italian]{babel}

\usepackage[latin1]{inputenc}

\usepackage[T1]{fontenc}

\usepackage{graphicx}

\usepackage{indentfirst}

\usepackage{amsmath,amssymb}

\usepackage{enumitem} 

\newcommand{\virgolette}[1]{``#1''}

\usepackage[margin=1in]{geometry} %Smaller margins

\usepackage{lmodern} %Vector PDF

\usepackage{siunitx}

\usepackage{xcolor}

\usepackage{colortbl}

\usepackage{multirow}

\usepackage{rotating}

\usepackage{booktabs}

\usepackage{longtable}

\newcommand*\chem[1]{\ensuremath{\mathrm{#1}}}

\begin{document}

\begin{titlepage}
	\centering
	{\scshape\LARGE Laboratorio di Ottica, Elettronica e \\ Fisica Moderna \par}
	\vspace{1cm}
	{\scshape\Large Gruppo Luned� 12\par}
	\vspace{1.5cm}
	{\huge\bfseries Spettrometro a prisma\par}
	\vspace{2cm}

	{\Large\itshape Nicol� Cavalleri, Giacomo Lini e Davide Passaro}

	\vspace{5cm}
	\vfill
	\begin{abstract}
	
		Di seguito vengono riportate ed esaminate le procedure compiute per la misura di diverse grandezze fisiche caratteristiche di un sistema composto da un reticolo che presenta fenomeni di interferenza e diffrazione. Nello specifico, dato un reticolo, viene determinato il passo, cio� la distanza tra due fenditure, a partire dallo spettro di emissione di una sostanza con lunghezza d'onda nota; vengono determinati anche il potere dispersivo e risolutivo, rispettivamente la distanza angolare tra due righe spettrali a un determinato ordine e un indice della capacit� dello strumento di risolvere in maniera precisa delle righe di dispersione. Data inoltre una lampada al mercurio (\chem{Hg}), ne viene determinato lo spettro di emissione, con le lunghezze d'onda caratteristiche. 
	
	\end{abstract}

	\vfill
	{\large \today\par}

\end{titlepage}

\newpage
		
\end{document}