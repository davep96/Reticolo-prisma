\documentclass[a4paper,11pt]{article}

\usepackage[italian]{babel}

\usepackage[latin1]{inputenc}

\usepackage[T1]{fontenc}

\usepackage{graphicx}

\usepackage{indentfirst}

\usepackage{amsmath,amssymb}

\usepackage{enumitem} 

\newcommand{\virgolette}[1]{``#1''}

\usepackage[margin=1in]{geometry} %Smaller margins

\usepackage{lmodern} %Vector PDF

\usepackage{siunitx}

\usepackage{xcolor}

\usepackage{colortbl}

\usepackage{multirow}

\usepackage{rotating}

\usepackage{booktabs}

\usepackage{longtable}

\newcommand*\chem[1]{\ensuremath{\mathrm{#1}}}

\begin{document}
	
	\section{Iter sperimentale}
	L'iter sperimentale eseguito per la valutazione dell'indice di rifrazione del prisma può essere suddiviso in tre parti:
	\begin{enumerate}
		\item Calibrazione dello spettroscopio.
		\item Valutazione dell'angolo del prisma.
		\item Valutazione degli angoli di deviazione minima per ciascuna lunghezza d'onda.
	\end{enumerate}
\end{document}