\documentclass[a4paper,11pt]{article}

\usepackage[italian]{babel}

\usepackage[latin1]{inputenc}

\usepackage[T1]{fontenc}

\usepackage{graphicx}

\usepackage{indentfirst}

\usepackage{amsmath,amssymb}

\usepackage{enumitem} 

\newcommand{\virgolette}[1]{``#1''}

\usepackage[margin=1in]{geometry} %Smaller margins

\usepackage{lmodern} %Vector PDF

\usepackage{siunitx}

\usepackage{xcolor}

\usepackage{colortbl}

\usepackage{multirow}

\usepackage{rotating}

\usepackage{booktabs}

\usepackage{longtable}

\newcommand*\chem[1]{\ensuremath{\mathrm{#1}}}

\begin{document}


\section{Strumentazione}
Per estrarre i valori relativi all'angolo del prisma e la deviazione minima � stata usata la seguente strumentazione:
\begin{description}[align=left]
	
	\item [Lampada al mercurio] Sorgente di luce di fascio noto studiato in precedenza. 
	
	\item [Prisma trasparente] Strumento utilizzato per permettere la rifrazione della luce. L'obiettivo finale dell'esperienza era quello di misurare l'indice di rifrazione del vetro che lo costituiva. Questo era a base triangolare ($60\deg$ circa ad occhio) e una delle facce laterali era coperta di nastro adesivo opaco.
	
	\item [Spettrometro] Principale strumento utilizzato. Lo spettrometro � servito per la misura degli angoli sottesi dai fasci di luce deviati dal prisma. Questo era composto da quattro parti:
	
	\begin{description} [align=left]
		\item [Collimatore] Componente utilizzata per collimare i raggi provenienti dalla lampada. Il collimatore era fisso alla base dello spettrometro, senza possibilit� di movimento. Per la messa a punto del collimatore erano presenti due viti. Una per regolare il fuoco del collimatore e una per variare l'apertura della fenditura dalla quale entrava la luce.
		\item [Piatto] Utilizzato come sostegno per il prisma. Il piatto era posto parallelamente al piano di lavoro ed era dotato di due pinze verticali per impedire il movimento al reticolo. Il piatto era regolabile in altezza e per rotazioni sul suo asse. Era dotato di tre viti, una di bloccaggio rispetto alle regolazioni in altezza, una di bloccaggio rispetto alle rotazioni e una vite micrometrica per piccole rotazioni sul suo asse.
		\item[Cannocchiale] Utilizzato per l'osservazione dei raggi luminosi diffratti dal prisma. Il cannocchiale era collegato allo spettrometro in modo che potesse girare intorno al piatto. Inoltre era dotato di tre viti: una per la regolazione del fuoco, una per il bloccaggio rispetto alle rotazioni intorno al piatto ed una vite micrometrica.
		\item[Goniometro] Utilizzato per la misura degli angoli, dotato di due coppie di noni contrapposti. Il goniometro possedeva una sensibilit� al terzo di grado (ossia venti primi), teoricamente estendibile tramite l'utilizzo dei noni al mezzo primo. Pi� realisticamente l'incertezza dovuta alla difficoltosa lettura del nonio contribuiva con un errore di circa un primo.
	\end{description}
\end{description}

\end{document}