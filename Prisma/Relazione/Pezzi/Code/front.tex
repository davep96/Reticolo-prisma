\documentclass[a4paper,11pt]{article}

\usepackage[italian]{babel}

\usepackage[latin1]{inputenc}

\usepackage[T1]{fontenc}

\usepackage{graphicx}

\usepackage{indentfirst}

\usepackage{amsmath,amssymb}

\usepackage{enumitem} 

\newcommand{\virgolette}[1]{``#1''}

\usepackage[margin=1in]{geometry} %Smaller margins

\usepackage{lmodern} %Vector PDF

\usepackage{siunitx}

\usepackage{xcolor}

\usepackage{colortbl}

\usepackage{multirow}

\usepackage{rotating}

\usepackage{booktabs}

\usepackage{longtable}

\newcommand*\chem[1]{\ensuremath{\mathrm{#1}}}

\begin{document}

\begin{titlepage}
	\centering
	{\scshape\LARGE Laboratorio di Ottica, Elettronica e \\ Fisica Moderna \par}
	\vspace{1cm}
	{\scshape\Large Relazione di Laboratorio 2\par}
	\vspace{1.5cm}
	{\huge\bfseries Spettrometro a prisma\par}
	\vspace{2cm}

	{\Large\itshape Nicol� Cavalleri, Giacomo Lini e Davide Passaro
		
	(LUN12)}

	\vspace{5cm}
	\vfill

	\begin{abstract}
		Vengono presentati in questo documento la procedura e l'analisi dati di un esperimento di ottica, volto a validare la legge teorica della dispersione cromatica di Cauchy. Lo strumento principale attorno cui si svolge l'intera relazione � il prisma, qui in vetro, utilizzato per rifrangere il raggio di luce.
	\end{abstract}


	\vfill
	{\large \today\par}

\end{titlepage}

\newpage
		
\end{document}