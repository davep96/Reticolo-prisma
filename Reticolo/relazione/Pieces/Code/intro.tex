\documentclass[a4paper,11pt]{article}

\usepackage[italian]{babel}
\usepackage[latin1]{inputenc}
\usepackage[T1]{fontenc}
\usepackage{graphicx}

\usepackage{amsmath,amssymb}
\usepackage{enumitem}
\usepackage{lmodern}
\usepackage{siunitx}
\newcommand*\chem[1]{\ensuremath{\mathrm{#1}}}

\newcommand{\virgolette}[1]{``#1''}

\author{Nicol� Cavalleri, Giacomo Lini, Davide Passaro}

\title{Spettrometro a Reticolo}

\begin{document}

\maketitle
\author

\begin{abstract}

	Di seguito vengono riportate ed esaminate le procedure compiute per la misura di diverse grandezze fisiche caratteristiche di un sistema composto da un reticolo che presenta fenomeni di interferenza e diffrazione. Nello specifico, dato un reticolo, viene determinato il passo, cio� la distanza tra due fenditure, a partire dallo spettro di emissione di una sostanza con lunghezza d'onda nota; vengono determinati anche il potere dispersivo e risolutivo, rispettivamente la distanza angolare tra due righe spettrali a un determinato ordine e un indice della capacit� dello strumento di risolvere in maniera precisa delle righe di dispersione. Data inoltre una lampada al mercurio (\chem{Hg}), ne viene determinato lo spettro di emissione, con le lunghezze d'onda caratteristiche. 
	
\end{abstract}
	
\section{Introduzione} 

	Uno spettrometro a reticolo � un sistema fisico composto da una base fissa su cui sono disposte due strutture rotanti, collegate a un goniometro e dei noni
	per la misura di angoli. Al centro di questa struttura si trova un sostegno dove viene posto il reticolo, vale a dire una lastra di vetro con delle fenditure molto numerose e sottili. A questa piattaforma sono poi collegati un collimatore, cio� un dispositivo che \virgolette{raddrizza} o \virgolette{collima}, il fascio di luce rendendolo perpendicolare al reticolo, e un cannocchiale che consente di osservare lo spettro di emissione della luce incidente il reticolo. Avvicinando il collimatore ad una
	lampada � possibile catturare la luce emessa dalla stessa e osservarne lo spettro di emissione. \\
	
	Dal punto di vista matematico la relazione fondamentale nell'analisi di uno spettro di emissione di una fonte luminosa � la seguente:
	\begin{equation} \label{master}
		d \, \sin \theta  = k \, \lambda
	\end{equation}
	dove $d$ rappresenta il passo del reticolo, cio� la distanza tra i punti medi di due fenditure vicine, $\theta$ l'angolo di deflessione del raggio di luce reso monocromatico dal reticolo, $k$ l'ordine dello spettro contenente il raggio considerato e $\lambda$ la lunghezza d'onda della radiazione luminosa emessa. Chiaramente a diverse lunghezze d'onda corrispondono anche diverse intensit� di emissione, che sperimentalmente si osservano in maniera intuitiva a partire dalla luminosit� delle righe di emissione. La relazione matematica che sta alla base di questo fenomeno � la seguente:
	\begin{equation} \label{intensity}
		I(\theta) \propto \dfrac{\sin^2 \, \left(m \, \frac{d}{\lambda} \, \pi \sin{\theta} \right)}{\sin^2 \, \left(\frac{d}{\lambda} \,\pi \sin{\theta}\right)}
	\end{equation}
	dove $ I $ rappresenta l'intensit� luminosa, $m$ il numero di fessure del reticolo che sono investite dalla luce e le altre costanti sono come per la formula $\left( \ref{master} \right)$. \\
	
	L'equazione $\left( \ref{master} \right)$, con gli strumenti a disposizione � caratterizzata dal fatto di avere due incognite, $\lambda$ e $d$. Per questa ragione l'esperimento si � svolto in due fasi: la determinazione del passo ($ d $) a partire da uno spettro noto, e la determinazione dello spettro una volta noto il passo. Nel nostro caso per la determinazione del passo � stata usata una lampada al sodio la cui emissione � caratterizzata da due lunghezze d'onda specifiche:
	\[ 
		\lambda_1 = \SI{589.0}{\nano\meter} \quad
		\lambda_2 = \SI{589.6}{\nano\meter}
	\]
	corrispondenti a due righe gialle distinte ma molto vicine nello spettro. Risolvendo $\left( \ref{master} \right)$ rispetto a $d$ � possibile dunque determinare il passo del reticolo, che da incognita diventa un termine noto, con errore associato. A questo punto la stessa relazione garantisce di poter determinare le lunghezze d'onda dello spettro di emissione di un diverso elemento chimico, nel caso in questione il mercurio (\chem{Hg}).

\end{document}